\chapter{Mobile Devices Resilient to Capture}
Quando un dispositivo mobile viene rubato, il proprietario non perde soltanto il valore economico del dispositivo ma anche i dati al suo interno. E' pertanto un problema non banale quello di tutelare gli utenti affinché dei dati "fisicamente" rubati, non possano essere acceduti.\\
Vorremmo creare uno schema che renda un device impossibile di essere usato senza il consenso del proprietario, ovvero:
\begin{definition}[Capture Resilient Device]\label{def:resdev}
Dispositivo che può essere usato \textbf{solo e soltanto} dal proprietario.
\end{definition}
\begin{remark}
Assumiamo che il \textbf{"core"} di questo dispositivo sia una secret-key.
\end{remark}
Alcuni approcci possibili possono essere:
\begin{itemize}
    \item password per bloccare/sbloccare (cifrare/decifrare) il dispositivo.
    \item proteggere il segreto in un hardware "tamper-proof" (hardware che non può essere manomesso).
    \item Download di una chiave da una repository in rete quando serve. 
\end{itemize}
Sappiamo che una password è una soluzione \textbf{generalmente} debole per proteggere un segreto, una repository remota (o il suo gestore) deve essere affidabile o potrebbero esserci altri problemi e un hardware \textbf{\textit{realmente}} impossibile da manomettere \textbf{per sempre} non è detto esista. Sembra quindi che siano necessarie altri sistemi.
\section{Capture-Protection Server}
MacKenzie e Raiter nel 2003 risolsero il problema, creando uno schema \textit{a tre player} che coinvolgeva un \textbf{untrusted server} che poteva essere anche non fidato e due approcci possibili: standard cryptography o secret sharing.
\begin{remark}
L'idea è che se un attaccante dovesse riuscire a rubare il segreto di uno dei tre attori o una combinazione di 2 attori su 3, \textbf{non} è \textbf{possibile} dedurre niente e rompere il sistema.
\end{remark}
\begin{remark}
C'è una combinazione che può rompere il sistema: se un attaccante riuscisse a rubare sia la password che il device, allora sarebbe "digitalmente identico" al proprietario e non ci sarebbe niente da fare. \textbf{A meno di usare secret sharing}
\end{remark}
\begin{definition}[Capture-Protection Server]\label{def:capserver}
Consideriamo:
\begin{itemize}
    \item \textbf{Utente} $U$, detiene \textbf{password} $P$.
    \item \textbf{Dispositivo} $D$, detiene $\textcolor{red}{SK_D}, PK_D$.
    \item \textbf{Server Remoto} $S$, detiene $SK_S, PK_S$.
\end{itemize}
Assumiamo che il \textbf{device} sia \textbf{sempre connesso} alla \textbf{rete} e che la sua chiave sia \textbf{protetta} da \textbf{cifratura} e che sia decifrabile \textbf{solo cooperando con il server}.\\
Il protocollo si divide in due fasi:
\end{definition}
\begin{definition}[Capture-Protection Server: Device Init]\label{def:devinit}
    \begin{enumerate}
        \item L'utente registra, temporaneamente, \textcolor{red}{$P$} nel device. 
        \item Scarichiamo, temporaneamente, sul device la \textcolor{red}{$PK_S$}.
        \item Generiamo \textbf{due chiavi random} $v,a$ e calcoliamo
        \[\textcolor{red}{b=H(P)}\]
        \item Tramite stream-cipher\footnotemark cifriamo la $SK_D$
        \[\textcolor{red}{c=SK_D\oplus PRF(v,P)}\]
        \footnotetext{\textsuperscript{\thefootnote}ad esempio AES-CTR (\cref{def:ctrmode})}
        Dove il keystream generato dalla $PRF$ dipende da un valore generato da $D$ e da un valore generato da $U$: \textbf{La ricostruzione è possibile solo se conosco entrambi}\footnotemark.
        \footnotetext{\textsuperscript{\thefootnote}Questa costruzione permette di evitare il secret-sharing poiché il funzionamento è analogo: genero un segreto a partire da due valori, senza i quali non è rigenerabile.}
        \item Generiamo un \textbf{ticket}\footnotemark tramite una funzione di \textbf{envelope}:
        \[tkt=E_{PK_S}[a,b,c]=[a, H(\text{real pwd}), SK_S\oplus \text{keystream}]\]
        e lo tengo in locale per un momento successivo.
        \footnotetext{\textsuperscript{\thefootnote}Poiché l'\textbf{envelope} è fatto in funzione della public-key del server, questo la potrà decriptare ma non verrà mai a conoscenza degli altri segreti poiché sono cifrati.}
        \item \textbf{Cancello} dal device: la password $P$ e il suo hash $b$, il ciphertext $c$ e la $SK_D$.
    \end{enumerate}
\end{definition}
\begin{definition}[Capture-Protection Server: Key Retrieval]\label{def:keyretr}
\begin{enumerate}
    \item \textbf{Nel device} troviamo: $v,a,tkt,PK_D$.
    \item L'utente \textbf{accede} al dispositivo con $P$.
    \item Calcoliamo l'hash della password \textbf{inserita}:\[\beta=H(P)\]
    \textbf{se l'utente è legittimo} $\beta\equiv b$.
    \item Genero un random $\rho$ per cifrare il canale di comunicazione.
    \item Invio al \textbf{remote-server} il $tkt$ e la sua autenticazione:
    \[\text{HMAC}_a[tkt, E_{PK_S}[\beta,\rho]]\]
    \item Il server decrypta il ticket e l'envelope, leggendo $a,b,c,\beta,\rho$.
    \item Il server \textbf{controlla l'HMAC ricevuto} utilizzando la chiave $a$ e \textbf{autentica il device}.
    \item Il server \textbf{decrypta l'hash della password} $\beta$ e la confronta con $b$ per \textbf{autenticare l'utente}.
    \item Il server invia $\rho\oplus c$, per proteggere il trasporto della secret-key (cifrata) del device.
    \item Il device ottiene la \textbf{sua} chiave privata decifrando:
    \[(\rho \oplus c)\oplus \rho \oplus PRF(v,P)=SK_D\]
    Solo un utente che ha il device e la password può decifrare la $SK_D$.
    \end{enumerate}
\end{definition}
\begin{remark}
Un attaccante in possesso del device potrebbe vedere soltanto i valori di $v,a$ e il $tkt$. Tuttavia, dovrebbe riuscire a rompere l'envelope e dopo invertire l'hash o rompere $c$.
\end{remark}
Le vulnerabilità aperte su questo schema sono due:
\begin{itemize}
    \item \textbf{L'attaccante viene in possesso di $SK_D$:} in questo caso è possibile ad esempio disconnettere il device dalla rete e accedere a tutto.
    \item \textbf{device rubato e password predicibile}: prima o poi sarà possibile accedere alla secre-key facendo brute-forcing.
\end{itemize}
Negli altri casi siamo protetti:
\begin{itemize}
    \item \textbf{Server Cracked AND Password Known}: l'attaccante non può firmare/decifrare perché $v$ è \textbf{tenuta segreta nel device}, quindi $SK_D$ è cifrata con $PRV(v,P)$ e non può essere ottenuta.
    \item \textbf{Device Cracked/Stolen}: l'attaccante può soltanto fare dictionary attack \textbf{online}. \textbf{TUTTAVIA} è facilmente individuabile perché anche se i \textbf{tentativi} di attacco sono \textbf{autenticati}, la password fallisce molte volte, a meno di avere una password difficile.
    \item \textbf{Device and Server Cracked:} l'attaccante può soltanto fare dictionary attack \textbf{offline} perché conosce $b$ e cercare la password.
\end{itemize}
La vulnerabilità rimasta al protocollo può essere fixata introducendo un \textbf{threshold secret-sharing}, \textbf{nascondendo} la SK anche allo \textbf{stesso device}.
%% finisci slide 31 aggiungendo versione boostata del protocollo. Qui sotto cancella tanto è inutile.
\begin{corollary}[Threshold SS for Capture-Protection: Device Init]\label{def:captureboost}
Supponiamo di utilizzare uno schema di Secret-Sharing (2,2) con RSA come sistema di firma:
\begin{enumerate}
        \item L'utente registra, temporaneamente, \textcolor{red}{$P$} nel device. 
        \item Scarichiamo, temporaneamente, sul device la \textcolor{red}{$PK_S$}.
        \item Generiamo \textbf{due chiavi random} $v,a$ e calcoliamo
        \[\textcolor{red}{b=H(P)}\]
        \item Assegniamo $n=p\cdot q$. Quindi \textcolor{red}{$\phi(n)=(p-1)(q-1)$} e seleziono $e,\textcolor{red}{d}$ secondo \textbf{RSA} (\cref{def:rsakey})
        \item Genero Share 1: \textcolor{red}{$d_1=PRF(v,P)$}.\footnotemark
        \footnotetext{\textsuperscript{\thefootnote}La ricostruzione dello share \textbf{a posteriori} è possibile \textbf{se solo se} conosco sia $v$ che $P$}
        \item Genero Share 2: \textcolor{red}{$d_2=d-d_1\mod\phi(n)$}.
        \item Genero \textbf{Disabling-Key}: $t$. Questa chiave \textbf{va mantenuta offline}.
        \item Genero \textbf{Disabling Value}: \textcolor{red}{$u=H(t)$}.
        \item Genero ed invio un \textbf{ticket} tramite una funzione di \textbf{envelope}:
        \[tkt=E_{PK_S}[a,b,u,d_2,N]\]
        \begin{itemize}
            \item $a$: device auth.
            \item $b$: pw auth.
            \item $d_2$: second share.
            \item $N$: RSA sign.
        \end{itemize}
        \item \textbf{Cancello} dal device tutto ciò che è in \textcolor{red}{rosso}.
    \end{enumerate}
\end{corollary}
\begin{corollary}[Threshold SS for Capture-Protection: Key Retrieval]\label{def:retrboost}
Sia $m$ il messaggio da firmare, \textbf{ogni volta} che avviene una transazione con il server.
\begin{enumerate}
    \item \textbf{Nel device} troviamo: $v,a,tkt,e,N$.
    \item L'utente \textbf{accede} al dispositivo con $P$.
    \item Calcoliamo l'hash della password \textbf{inserita}:\[\beta=H(P)\]
    \textbf{se l'utente è legittimo} $\beta\equiv b$.
    \item Genero un random $\rho$ per cifrare il canale di comunicazione.
    \item Invio al \textbf{remote-server} il $tkt$ e la sua autenticazione:
    \[\text{HMAC}_a[tkt, E_{PK_S}[H(m),\beta,\rho]]\]
    \item Il server decrypta il ticket, leggendo $a,b,d_2,u,N$.
    \item Il server \textbf{controlla l'HMAC ricevuto} utilizzando la chiave $a$ e \textbf{autentica il device}.
    \item \textbf{Se solo se} il \textbf{device} è \textbf{autenticato} il server \textbf{decrypta il resto del messaggio} leggendo $H[m]\beta,\rho$.
    \item Il server confronta $\beta$ con $b$ per \textbf{autenticare l'utente}.
    \item Il server controlla se $u\ne{H(t)}$, dove $t$ è una delle \textbf{disabling-key} mantenute nella black-list del server. \textbf{Se c'è un match}, allora il \textbf{device} è stato \textbf{rubato} e ignora il resto del protocollo, altrimenti...
    \item Il server invia la sua firma del messaggio $\rho\oplus H[m]^{d_2}$, per proteggere il trasporto della secret-key (cifrata) del device.
    \item Il device decifra il messaggio ottenuto per ottenere \[H[m]^{d_2}=(\rho\oplus H[m]^{d_2})\oplus \rho\]
    \item Il device \textbf{completa la firma}, calcolando a runtime:
    \begin{align*}
        d_1=PRF[v,P]\\
        H[m]^{d_2}H[m]^{d_1}\bmod{N}
    \end{align*}
    \end{enumerate}
\end{corollary}
\begin{remark}
La \textbf{signature-key} $d$ \textbf{non è mai} trasmessa in chiaro. Pertanto, se un attaccante ottiene \textbf{sia} la password \textbf{che} il device, non può ottenere $d$ e firmare messagi come fosse il proprietario.
\end{remark}
Con questo metodo, possiamo ampliare il protocollo con una funzione di disattivazione:
\begin{corollary}[Threshold SS for Capture-Protection: Key Disabling]\label{def:keydis}
Il server mantiene un database dove salva le associazioni $\{ticket,t\}$ che vengono black-listate. Se un utente \textbf{mantiene una copia off-line} di $t$, e il suo device viene rubato, può inviare al server in un qualsiasi momento un messaggio contenente la disabling-key e il server dopo aver autenticato l'utente inserirà il device nella black-list.
\end{corollary}
